\documentclass{article}

% PACKAGES
\usepackage[margin=0.75in]{geometry}
\usepackage{amsmath}
\usepackage{amssymb}
\usepackage{graphicx}
\usepackage{subcaption}
\usepackage{braket}
\usepackage{nopageno}
\usepackage{dsfont}
\usepackage{bbold}

\newcommand{\assignment}[1]{
    \newpage
    \begin{tabular}{p{0.65\linewidth}p{0.25\linewidth}}
        {\bf\LARGE Physics 212B - Homework #1 }
        &
        \parbox[b]{0.24\textwidth}{
            \hfill Eric Weise

            \hfill A09642187
            }
    \end{tabular}
    \vspace{12pt}
    \newline
}
\newcommand{\D}[1]{{d#1 \,}}
\newcommand{\Poperator}[0]{-i \hbar \partial_x}
\renewcommand{\exp}[1]{\,\text{exp} \big\{ #1 \big\} }


\begin{document}

\assignment{1}
For a non-relativistic free particle \( L = \frac{m\dot{x}}{2} \)
\subsection*{Part 1}
Show that the stationary (classical) action $S[x]$ corresponding to the
classical motion of a free particle travelling from $(x_0,t_0)$ to
$(x_1,t_1)$ is \( S[x]= \frac{m(x_1-x_0)^2}{2(t_1-t_0)}\)

\begin{align*}
    S[x]
    &= \int_{t_0}^{t_1} \frac{m}{2}\dot{x}^2 dt \\
    &= \frac{m}{2} \int_{t_0}^{t_1} v(t) v(t) dt \\
    &= \frac{m}{2} \Big[ \big[ v(t) x(t) \big]_{t_0}^{t_1} - \int_{t_0}^{t_1} x(t) dv(t) \Big]
        && \text{Using itegration by parts.} \\
    &= \frac{m}{2} \Big[ \big[ v(t) x(t) \big]_{t_0}^{t_1} - \int_{t_0}^{t_1} x(t) \ddot{x}(t) dt \\
    &= \frac{m}{2} \big[ v(t) x(t) \big]_{t_0}^{t_1}
        && \text{since $\ddot{x}=0$ (Free particle)} \\
    &= \frac{m}{2} \big[ v(t_1) x(t_1) - v(t_0) x(t_0) \big] \\
    &= \frac{m}{2} \big[ \frac{x_1 - x_0}{t_1 - t_0} x(t_1) - \frac{x_1 - x_0}{t_1 - t_0} x(t_0) \big] \\
    &= \frac{m}{2} \frac{(x_1 - x_0)^2}{t_1 - t_0}
\end{align*}

\subsection*{Part 2}
Show that the spatial derivative of the action \( \partial_{x_1} S[x] \) is the momentum of the particle.

\begin{align*}
    \partial_{x1} S[x]
    &= \partial_{x_1} \frac{m}{2} \frac{(x_1 - x_0)^2}{t_1 - t_0} \\
    &= \frac{m}{2} \frac{2(x_1 - x_0)(\partial_{x_1}(x_1 - x_0))}{t_1 - t_0}
        && \text{By the chain rule} \\
    &= \frac{m}{2} \frac{2(x_1 - x_0)}{t_1 - t_0} \\
    &= m \frac{x_1 - x_0}{t_1 - t_0} \\
    &= m \cdot v
\end{align*}

\subsection*{Part 3}
Show that the (negative) temporal derivative of the action, \( -\partial_{t_1} S[x] \), is the energy of the particle.

\begin{align*}
    -\partial_{t_1} S[x]
    &= -\partial_{t_1} \frac{m}{2} \frac{(x_1 - x_0)^2}{t_1 - t_0} \\
    &= - \frac{m}{2} (x_1 - x_0)^2 \big[ \partial_{t_1} (t_1 - t_0)^{-1} \big] \\
    &= - \frac{m}{2} (x_1 - x_0)^2 \big[ (-1) (t_1 - t_0)^{-2} \big] \\
    &= \frac{m}{2} \big(\frac{x_1 - x_0}{t_1 - t_0}\big)^2 \\
    &= \frac{m}{2} \ddot{x}^2
\end{align*}


%%%%%%%%%%%%%%%%%%%%%%
%%%%%%%% HW 2 %%%%%%%%
%%%%%%%%%%%%%%%%%%%%%%

\assignment{2}

Consider state
\( \ket{\psi} = \int \D{x} \, \psi(x) \ket{x} \)
described by the following wave function with a tunable parameter $\sigma$:
\[ \psi(x) = \frac{1}{\pi^{1/4} \sigma^{1/2}} \exp\big( -\frac{x^2}{2\sigma^2} \big) \]

\subsection*{Part 1}
Check that the state is normalized.

\begin{align*}
    \braket{\psi|\psi}
    & = \int_{-\infty}^{\infty} \D{x} \frac{1}{\pi^{1/4} \sigma^{1/2}} \exp\Big( \frac{-x^2}{2\sigma^2} \Big)
        \frac{1}{\pi^{1/4} \sigma^{1/2}} \exp\Big( \frac{-x^2}{2\sigma^2} \Big) \\
    & = \frac{1}{\sqrt{\pi} \sigma} \int_{-\infty}^{\infty} \D{x} \exp\Big( \frac{-x^2}{\sigma^2} \Big) \\
    \big| \braket{\psi|\psi} \big| ^2
    & = \bigg( \frac{1}{\sqrt{\pi} \sigma} \int_{-\infty}^{\infty} \D{x} \exp\Big( \frac{-x^2}{\sigma^2} \Big) \bigg)
        \bigg( \frac{1}{\sqrt{\pi} \sigma} \int_{-\infty}^{\infty} \D{y} \exp\Big( \frac{-y^2}{\sigma^2} \Big) \bigg) \\
    & = \frac{1}{\pi\sigma^2} \iint_{-\infty}^{\infty} \D{x} \D{y} \exp\big( -\frac{x^2 + y^2}{\sigma^2} \big) \bigg) \\
    & = \frac{1}{\pi\sigma^2} \int_{0}^{2\pi}\D{\theta} \int_{0}^{\infty} \D{r} r \exp\Big( \frac{-r^2}{\sigma^2} \Big)
        & \text{Change of to polar coordinates}\\
    & = \frac{2\pi}{\pi\sigma^2} \int_{0}^{\infty} \D{r} r \exp\Big( \frac{-r^2}{\sigma^2} \Big) \\
    & = \int_{0}^{\infty} \D{r} \frac{2r}{\sigma^2} \exp\Big( \frac{-r^2}{\sigma^2} \Big) \\
    & = -\int_{0}^{-\infty} \D{s} \exp{s}
        & \text{Changing Coordinates: \( s=\frac{-r^2}{\sigma^2} \) and \( \D{s} = \frac{-2r}{\sigma^2} \D{r} \)} \\
    & = -\exp{-\infty} + \exp0 \\
    & = 1 \\
    \big| \braket{\psi|\psi} \big| ^2 &= 1 \implies \braket{\psi|\psi} = 1
\end{align*}


\subsection*{Part 2}
Evaluate the expectation values:
\( \braket{\hat{x}} \),
\( \braket{\hat{p}} \),
\( \braket{\hat{x}^2} \), and
\( \braket{\hat{p}^2} \)
in terms of $\sigma$.

\begin{align*}
    \braket{\hat{x}}
    &= \braket{\psi|\hat{x}|\psi} \\
    &= \int \D{x}\D{y}\D{z}
        \frac{1}{\pi^{1/4}\sigma^{1/2}} \exp{\frac{-y^2}{2\sigma^2}}
        \braket{y|x}x\braket{x|z}
        \frac{1}{\pi^{1/4}\sigma^{1/2}} \exp{\frac{-z^2}{2\sigma^2}} \\
    &= \frac{1}{\sqrt{\pi}\sigma} \int \D{x}\D{y}\D{z}
        \delta(y-x)\delta(x-z) x \, \exp{\frac{-y^2-z^2}{2\sigma^2}} \\
    &= \frac{1}{\sqrt{\pi}\sigma} \int \D{x} x \, \exp{\frac{x^2}{\sigma^2}} \\
    &= \frac{1}{\sqrt{\pi}\sigma} \Big[ 
        \int_{-\infty}^{0} \D{x} x \, \exp{\frac{-x^2}{\sigma^2}} +
        \int_{0}^{\infty} \D{x} x \, \exp{\frac{-x^2}{\sigma^2}} \Big] \\
    &= \frac{1}{\sqrt{\pi}\sigma} \Big[ 
        \int_{\infty}^{0} (-\D{x}) (-x) \, \exp{\frac{-(-x)^2}{\sigma^2}} +
        \int_{0}^{\infty} \D{x} x \, \exp{\frac{-x^2}{\sigma^2}} \Big]
        && \text{Change variables from $x$ to $-x$ in first integral} \\
    &= \frac{1}{\sqrt{\pi}\sigma} \Big[ 
        - \int^{\infty}_{0} x \, \exp{\frac{-x^2}{\sigma^2}} +
        \int_{0}^{\infty} x \, \exp{\frac{-x^2}{\sigma^2}} \Big]
        && \text{Swap integration limits} \\
    &= 0
\end{align*}

\begin{align*}
    \braket{\hat{x}^2}
    &= \int \D{w}\D{x}\D{y}\D{z}
        \frac{1}{\pi^{1/4}\sigma^{1/2}} \exp{\frac{-w^2}{2\sigma^2}}
        \braket{w|x}x\braket{x|y}y\braket{y|z}
        \frac{1}{\pi^{1/4}\sigma^{1/2}} \exp{\frac{-z^2}{2\sigma^2}} \\
    &= \frac{1}{\sqrt{\pi}\sigma} \int \D{x}\D{y}\D{z}
        \delta(w-x)\delta(x-y)\delta(y-z) xy \, \exp{\frac{-w^2-z^2}{2\sigma^2}} \\
    &= \frac{1}{\sqrt{\pi}\sigma} \int_{-\infty}^{\infty} \D{x} x^2 \, \exp{\frac{-x^2}{\sigma^2}} \\
    &= \frac{2}{\sqrt{\pi}\sigma} \int_{0}^{\infty} \D{x} x^2 \, \exp{\frac{-x^2}{\sigma^2}} 
        && \text{Integrand is symmetric around 0} \\
    &= \frac{\sigma^2}{\sqrt{\pi}} \int_{0}^{\infty} \Big( \D{x} \frac{2x}{\sigma^2} \big) \Big( \frac{x}{\sigma} \Big) \exp{\frac{-x^2}{\sigma^2}} \\
    &= \frac{\sigma^2}{\sqrt{\pi}} \int_{0}^{\infty} \D{t} t^{1/2} \exp{-t}
        && t = \frac{x^2}{\sigma^2} \text{ and } \D{t} = \frac{2x}{\sigma^2}\D{x}\\
    &= \frac{\sigma^2}{\sqrt{\pi}} \Gamma(\frac{3}{2}) \\
    &= \frac{\sigma^2}{\sqrt{\pi}} \frac{\sqrt{\pi}}{2} \\
    &= \frac{\sigma^2}{2}
\end{align*}

\begin{align*}
    \braket{\hat{p}}
    &= \braket{\psi|\Poperator|\psi} \\
    &= \frac{-i \hbar}{\sqrt{\pi}\sigma} \int \D{x} \exp{\frac{-x^2}{2\sigma^2}} \partial_x \exp{\frac{-x^2}{2\sigma^2}} \\
    &= \frac{-i \hbar}{\sqrt{\pi}\sigma} \int \D{x} \frac{-2x}{\sigma^2} \exp{\frac{-x^2}{\sigma^2}} \\
    &= \frac{2 i \hbar}{\sqrt{\pi}\sigma^3} \int \D{x} x \exp{\frac{-x^2}{\sigma^2}} \\
    &= \frac{2 i \hbar}{\sqrt{\pi}\sigma^3} \bigg[ \int_{-\infty}^{0} \D{x} x \exp{\frac{-x^2}{\sigma^2}} + \int_{0}^{\infty} \D{x} x \exp{\frac{-x^2}{\sigma^2}} \bigg] \\
    &= \frac{2 i \hbar}{\sqrt{\pi}\sigma^3} \bigg[ \int_{-\infty}^{0} \D{x} x \exp{\frac{-x^2}{\sigma^2}} + \int_{0}^{\infty} \D{x} x \exp{\frac{-x^2}{\sigma^2}} \bigg] \\
    &= \frac{i \hbar}{\sqrt{\pi}\sigma} \bigg[ \int_{+\infty}^{0} \D{u} \exp{-u} + \int_{0}^{\infty} \D{u} \exp{-u} \bigg]
        && u = \frac{x^2}{\sigma^2} \text{ and } \D{u} = \D{x}\frac{2x}{\sigma^2} \\
    &= \frac{i \hbar}{\sqrt{\pi}\sigma} \bigg[ -\int^{\infty}_{0} \D{u} \exp{-u} + \int_{0}^{\infty} \D{u} \exp{-u} \bigg] \\
    &= 0
\end{align*}

\begin{align*}
    \braket{\hat{p}^2}
    &= \braket{\psi|(\Poperator)(\Poperator)|\psi} \\
    &= \frac{(-i \hbar)^2}{\sqrt{\pi}\sigma} \int \D{x} \exp{\frac{-x^2}{2\sigma^2}} \partial_x \partial_x \exp{\frac{-x^2}{2\sigma^2}} \\
    &= \frac{-\hbar^2}{\sqrt{\pi}\sigma} \int \D{x} \exp{\frac{-x^2}{2\sigma^2}} \partial_x \Big( \frac{-x}{\sigma^2} \exp{\frac{-x^2}{2\sigma^2}} \Big) \\
    &= \frac{\hbar^2}{\sqrt{\pi}\sigma^3} \int \D{x} \exp{\frac{-x^2}{2\sigma^2}} \partial_x \Big( x \exp{\frac{-x^2}{2\sigma^2}} \Big) \\
    &= \frac{\hbar^2}{\sqrt{\pi}\sigma^3} \int \D{x} \exp{\frac{-x^2}{2\sigma^2}} \Big[ 1 - \frac{x^2}{\sigma^2} \Big] \exp{\frac{-x^2}{2\sigma^2}} \\
    &= \frac{\hbar^2}{\sqrt{\pi}\sigma^3} \int \D{x} \Big[ 1 - \frac{x^2}{\sigma^2} \Big] \exp{\frac{-x^2}{\sigma^2}} \\
    &= \frac{\hbar^2}{\sqrt{\pi}\sigma^3} \int \D{x} \exp{\frac{-x^2}{\sigma^2}} 
        - \frac{\hbar^2}{\sqrt{\pi}\sigma^3} \int \D{x} \frac{x^2}{\sigma^2} \exp{\frac{-x^2}{2\sigma^2}} \\
    &= \frac{\hbar^2}{\sqrt{\pi}\sigma^3} \Big[ \sqrt{\pi}\sigma \Big]
        - \frac{\hbar^2}{\sqrt{\pi}\sigma^3} \int \D{x} \frac{x^2}{\sigma^2} \exp{\frac{-x^2}{2\sigma^2}}
        && \text{Gaussian integral (same as part 1)} \\
    &= \frac{\hbar^2}{\sigma^2}
        - \frac{\hbar^2}{\sigma^4} \Big[ \frac{1}{\sqrt{\pi}\sigma} \int \D{x} x^2 \exp{\frac{-x^2}{2\sigma^2}} \Big] \\
    &= \frac{\hbar^2}{\sigma^2}
        - \frac{\hbar^2}{\sigma^4} \big[ i\frac{\sigma^2}{2} \big]
        && \text{Same as \(\braket{\hat{x}^2}\)}\\
    &= \frac{\hbar^2}{2\sigma^2}
\end{align*}


\subsection*{Part 3}
Based on the result of Part 2, calculate (stdx) and (stdp) in terms of $\sigma$.
Do they satisfy the uncertainty relation?

\begin{align*}
    (stdx) 
    &= \sqrt{\braket{\hat{x}^2} - \braket{\hat{x}}^2} \\
    &= \sqrt{\frac{\sigma^2}{2} - 0} \\
    &= \frac{\sigma}{\sqrt{2}}
\end{align*}

\begin{align*}
    (stdp) 
    &= \sqrt{\braket{\hat{p}^2} - \braket{\hat{p}}^2} \\
    &= \sqrt{\frac{\hbar^2}{2\sigma^2} - 0} \\
    &= \frac{\hbar}{\sqrt{2}\sigma}
\end{align*}

\[ (stdx)(stdp)
    = \frac{\sigma}{\sqrt{2}} \cdot \frac{\hbar}{\sqrt{2}\sigma}
    = \frac{\hbar}{2} 
    \ge \frac{\hbar}{2} \]

So, the uncertainty is satisfied (barely).


%%%%%%%%%%%%%%%%%%%%%%
%%%%%%%% HW 3 %%%%%%%%
%%%%%%%%%%%%%%%%%%%%%%
\assignment{3}
\renewcommand{\exp}[1]{ \, e^{#1} \, }
\newcommand{\aop}[0]{ \hat{a} \, }
\newcommand{\adagger}[0]{ \hat{a}^{\dagger} \, }

Consider
\( \hat{H} = \frac{1}{2}\big(\hat{p}^2 + \hat{x}^2\big) \),
derive the Heisenberg equation for operator
\( \aop = \frac{1}{\sqrt{2}} \big( \hat{x} + i \hat{p} \big) \)

 \subsection*{Operator Relationships}
 Define 
 \( \adagger = \frac{1}{\sqrt{2}} \big( \hat{x} - i \hat{p} \big)\).
 Then the following are true:

 \begin{tabular}{cc}
     \parbox[b]{0.45\textwidth}{
         \begin{align*}
             \aop\adagger
             &= \frac{1}{2} \big( \hat{x}^2 - i \hat{x}\hat{p} + i \hat{p}\hat{x} + \hat{p}^2 \big) \\
             &= \frac{1}{2} \big( \hat{x}^2  + \hat{p}^2 -i [\hat{x},\hat{p}] \big) \\
             &= \frac{1}{2} \big( \hat{x}^2  + \hat{p}^2 + \hpbar \big) \\
             &= \hat{H} + \frac{\hbar}{2} \\
         \end{align*}
     }
    &
    \parbox[b]{0.45\textwidth}{ 
         \begin{align*}
             \adagger\aop
             &= \frac{1}{2} \big( \hat{x}^2 - i \hat{p}\hat{x} + i \hat{x}\hat{p} + \hat{p}^2 \big) \\
             &= \frac{1}{2} \big( \hat{x}^2  + \hat{p}^2 +i [\hat{x},\hat{p}] \big) \\
             &= \frac{1}{2} \big( \hat{x}^2  + \hat{p}^2 - \hpbar \big) \\
             &= \hat{H} - \frac{\hbar}{2} \\
         \end{align*}
     }
 \end{tabular}

So
\( \hat{H}
    = \aop\adagger - \frac{\hbar}{2}
    = \adagger\aop + \frac{\hbar}{2}
\)

\subsection*{Deriving  Heisenberg's Equation}
\( \aop(t) = \exp{i\hat{H}t/\hbar} \aop \exp{-i\hat{H}t/\hbar} \)

\begin{align*}
    \frac{d}{dt} \aop(t)
    &= \frac{i}{\hbar}\hat{H} \exp{i\hat{H}t/\hbar} \aop \exp{-i\hat{H}t/\hbar} +
        \exp{i\hat{H}t/\hbar} \frac{\partial \aop}{\partial t} \exp{-i\hat{H}t/\hbar} - \frac{i}{\hbar} \exp{i\hat{H}t/\hbar} \aop \hat{H} \exp{-i\hat{H}t/\hbar} \\
    &= \frac{i}{\hbar} \exp{i\hat{H}t/\hbar} \Big( \hat{H}\aop - \aop\hat{H} \Big) \exp{-i\hat{H}t/\hbar} +
        \exp{i\hat{H}t/\hbar} \frac{\partial \aop}{\partial t} \exp{-i\hat{H}t/\hbar} \\
    &= \frac{i}{\hbar} \exp{i\hat{H}t/\hbar} \Big( \hat{H}\aop - \aop\hat{H} \Big) \exp{-i\hat{H}t/\hbar}
        && \text{since $\hat{x}$ and $\hat{p}$ are time independent.}\\
    &= \frac{i}{\hbar} \exp{i\hat{H}t/\hbar} \Big((\aop\adagger - \frac{\hbar}{2}) \aop - \aop(\adagger\aop + \frac{\hbar}{2}) \Big) \exp{-i\hat{H}t/\hbar} \\
    &= \frac{i}{\hbar} \exp{i\hat{H}t/\hbar} \Big(\aop\adagger\aop - \aop\adagger\aop - \hbar\aop \Big) \exp{-i\hat{H}t/\hbar} \\
    &= -i \, \exp{i\hat{H}t/\hbar} \big( \aop \big) \exp{-i\hat{H}t/\hbar} \\
    &= -i \, \aop(t)
\end{align*}


%%%%%%%%%%%%%%%%%%%%%%
%%%%%%%% HW 4 %%%%%%%%
%%%%%%%%%%%%%%%%%%%%%%
\assignment{4}
\newcommand{\xopr}{ \, \hat{x} \,}
\newcommand{\popr}{ \, \hat{p} \,}

\subsection*{Part 1}
Show that
\( [ \xopr, \popr^n ] = i \hbar n \popr^{n-1} \)
for
\( n \in \mathds{N} \)

Suppose that
\( [\xopr,\popr^n] = i\hbar n \popr^{n-1} \).
Then:
\begin{align*}
    [ \xopr, \popr^{n+1} ]
    &= \xopr\popr^{n+1} - \popr^{n+1}\xopr \\
    &= \xopr\popr^{n+1} - \popr\xopr\popr^n + \popr\xopr\popr^n - \popr^{n+1}\xopr \\
    &= (\xopr\popr-\popr\xopr)\popr^n + \popr(\xopr\popr^n - \popr^n\xopr) \\
    &= [\xopr,\popr]\popr^n + \popr[\xopr,\popr^n] \\
    &= i\hbar \popr^n + \popr i\hbar n \popr^{n-1} \\
    &= i\hbar \popr^n + i\hbar n \popr^n \\
    &= i\hbar (n+1) \popr^n
\end{align*}

\subsection*{Part 2}
Show that
\( [\xopr, F(\popr)] = i\hbar\partial_{\popr} F(\popr) \)
for generic function $F$.

Use
\[ F(\popr) = \sum_{n=0}^{\infty} \frac{f_n}{n!} \popr^n \]

then:
\begin{align*}
    [\xopr, F(\popr)]
    &= \sum_{n=0}^{\infty} \frac{f_n}{n!} [\xopr, \popr^n ] \\
    &= [\xopr, \mathbb{1}]  + \sum_{n=1}^{\infty} \frac{f_n}{n!} [\xopr, \popr^n ] \\
    &= \sum_{n=1}^{\infty} \frac{f_n}{n!} [\xopr, \popr^n ] \\
    &= \sum_{n=1}^{\infty} \frac{f_n}{n!} i\hbar n \popr^{n-1} \\
    &= i\hbar \sum_{n=1}^{\infty} \frac{f_n}{(n-1)!} i\hbar \popr^{n-1} \\
    &= i\hbar \sum_{n=0}^{\infty} \frac{f_{n+1}}{n!} \popr^{n} \\
    &= i\hbar \partial_{\popr} F(\popr)
\end{align*}

\subsection*{Part 3}
Show that
\( [\xopr, \hat{T}(a) ] = -a \hat{T}(a) \).

\begin{align*}
    [\xopr, \hat{T}(a) ]
    &= i\hbar \partial_{\popr} \exp{i\popr a / \hbar} \\
    &= i\hbar \frac{i a}{\hbar} \exp{i\popr a / \hbar} \\
    &= -a \exp{i\popr a / \hbar} \\
    &= -a \hat{T}(a)
\end{align*}


%%%%%%%%%%%%%%%%%%%%%%
%%%%%%%% HW 5 %%%%%%%%
%%%%%%%%%%%%%%%%%%%%%%
\assignment{5}
Show that
\( \ket{N} = \frac{1}{\sqrt{2\pi}} \int \D{\theta} \exp{i N \theta} \ket{\theta} \)
is normalized

\begin{align*}
    \braket{N|N}
    &= \frac{1}{2\pi} \int \D{\theta_1}\D{\theta_2} \exp{-i N \theta_1} \exp{i N \theta_2} \braket{\theta_1|\theta_2} \\
    &= \frac{1}{2\pi} \int \D{\theta} \exp{i N (\theta_2 - \theta_1)} \delta(\theta_2 - \theta_1) \\
    &= \frac{1}{2\pi} \int \D{\theta_1} \exp{0} \\
    &= \frac{1}{2\pi} \int_{0}^{2\pi} \D{\theta_1} \\
    &= \frac{1}{2\pi} 2\pi \\
    &= 1
\end{align*}


%%%%%%%%%%%%%%%%%%%%%%
%%%%%%%% HW 6 %%%%%%%%
%%%%%%%%%%%%%%%%%%%%%%
\assignment{6}
\subsection*{part 1}
From equation 149:
\[ \ket{\theta} = \frac{1}{\sqrt{2\pi}} \sum \exp{iN\theta} \ket{N} \]
\[ \ket{-\theta} = \frac{1}{\sqrt{2\pi}} \sum \exp{-iN\theta} \ket{N} \]
\[ \bra{\theta} = \frac{1}{\sqrt{2\pi}} \sum \exp{-iN\theta} \bra{N} \]
Then
\begin{align*}
    \int \ket{-\theta}\bra{\theta}
    &= \frac{1}{2\pi} \int \D{\theta} \Big( \sum_M\exp{iM\theta}\ket{M} \Big) \Big( \sum_N \exp{iN\theta}\bra{N} \Big) \\
    &= \frac{1}{2\pi} \int \D{\theta} \sum_{M,N}\exp{i(M+N)\theta}\ket{M}\bra{N} \\
    &= \frac{1}{2\pi} \Big( \int \D{\theta} \sum_{M=-N}\ket{M}\bra{N} + \int \D{\theta} \sum_{M\neq-N} \exp{i(M+N)\theta}\ket{N}\bra{M} \Big) \\
    &= \frac{1}{2\pi} \Big( 2\pi \sum_{n\in\mathbb{Z}} \ket{-N}\bra{N} + \sum_{M\neq-N} \Big[ \exp{i(M+N)\theta}\Big]_{\theta=0}^{2\pi} \ket{N}\bra{M}  \Big) \\
    &= \sum_{n\in\mathbb{Z}} \ket{-N}\bra{N} + 0
\end{align*}

\subsection*{part 2}
Show that $\hat{P}$ commutes with $\hat{H}$.

\begin{align*}
    \hat{H}\hat{P}
    &= \frac{1}{2}\hat{N}^2\sum\ket{-N}\bra{N} \\
    &= \frac{1}{2} \sum \hat{N}^2\ket{-N}\bra{N} \\
    &= \frac{1}{2} \sum (-N)^2\ket{-N}\bra{N} \\
    &= \frac{1}{2} \sum N^2\ket{-N}\bra{N}
\end{align*}
and
\begin{align*}
    \hat{P}\hat{H}
    &= \frac{1}{2} \sum \ket{-N}\bra{N} \hat{N}^2 \\
    &= \frac{1}{2} \sum \ket{-N}\bra{N} (N)^2 \\
    &= \hat{H}\hat{P}
\end{align*}


%%%%%%%%%%%%%%%%%%%%%%
%%%%%%%% HW 7 %%%%%%%%
%%%%%%%%%%%%%%%%%%%%%%
\assignment{7}
Find the expectation value of the raising operator as a function of $t$.

\begin{align*}
    \ket{\psi(0)}
    &= \frac{1}{\sqrt{4\pi}} \int \D{\theta} (1 + \exp{i\theta}) \ket{\theta} \\
    &= \frac{1}{\sqrt{4\pi}} \bigg( \int \D{\theta} \ket{\theta} + \int \D{\theta} \exp{i\theta} \ket{\theta} \bigg)
\end{align*}

Converting this to the $\ket{N}$ basis:
\begin{align*}
   \int \D{\theta} \exp{i\theta} \ket{\theta}
    &= \int \D{\theta} \exp{i1\theta} \ket{\theta} \\
    &=  \sqrt{2\pi}\ket{1} \\
\end{align*}
\begin{align*}
    \int \D{\theta} \ket{\theta}
    &= \int \D{\theta} \bigg( \frac{1}{\sqrt{2\pi}} \sum \exp{-iN\theta}\ket{N} \bigg) \\
    &= \frac{1}{\sqrt{2\pi}} \bigg[ \int \D{\theta} \ket{0} + \int \D{\theta} \sum_{N\neq0} \exp{-iN\theta}\ket{N} \bigg] \\
    &= \frac{1}{\sqrt{2\pi}} \bigg[ 2\pi \ket{0} + 0 \bigg]
        && \text{since, for nonzero $N$, $\int \D{\theta}\exp{-iN\theta} = 0$} \\
    &= \sqrt{2\pi}\ket{0}
\end{align*}

Then, in the momentum basis $\ket{N}$ the original wave function has the form:
\[ \ket{\psi(0)} = \frac{\sqrt{2\pi}}{\sqrt{4\pi}} ( \ket{0} + \ket{1} ) = \frac{1}{\sqrt{2}}(\ket{0} + \ket{1} ) \]

Now, applying the time evolution to find the wave function as a function of time:
\begin{align*}
    \ket{\psi(t)}
    &= \exp{-i\hat{H}t}\ket{\psi(0)} \\
    &= \frac{1}{\sqrt{2}} \exp{-i\hat{N}^2t/2}\big[ \ket{0} + \ket{1} \big] \\
    &= \frac{1}{\sqrt{2}} \big[ \exp{-i(0)^2t/2} \ket{0} + \exp{-i(1)^2t/2} \ket{1} \big] \\
    &= \frac{1}{\sqrt{2}} \big[ \ket{0} + \exp{-it/2} \ket{1} \big]
\end{align*}

Now, finding the expectation value of the raising operator as a function of time:
\begin{align*}
    \bra{\psi(t)} \exp{i\hat{\theta}} \ket{\psi(t)}
    &= \frac{1}{2} \big[ \bra{0} + \exp{it/2} \bra{1} \big]
        \exp{i\hat{\theta}}
        \big[ \ket{0} + \exp{-it/2} \ket{1} \big] \\
    &= \frac{1}{2} \big[ \bra{0} + \exp{it/2} \bra{1} \big]
        \big[ \ket{1} + \exp{-it/2} \ket{2} \big] \\
    &= \frac{1}{2} \big[
        \braket{0|1}
        + \exp{-it/2} \braket{0|2}
        + \exp{it/2} \braket{1|1}
        + \braket{1|2}
        \big] \\
    &= \frac{1}{2} \exp{it/2}
\end{align*}

\end{document}
