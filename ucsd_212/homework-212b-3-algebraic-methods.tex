\documentclass{article}

% PACKAGES
\usepackage[margin=0.75in]{geometry}
\usepackage{amsmath}
\usepackage{amssymb}
\usepackage{graphicx}
\usepackage{subcaption}
\usepackage{braket}
\usepackage{nopageno}
\usepackage{dsfont}
\usepackage{bbold}

\newcommand{\assignment}[1]{
    \newpage
    \begin{tabular}{p{0.65\linewidth}p{0.25\linewidth}}
        {\bf\LARGE Physics 212B - Part 3 - Homework #1 }
        &
        \parbox[b]{0.24\textwidth}{
            \hfill Eric Weise

            \hfill A09642187
            }
    \end{tabular}
    \vspace{12pt}
    \newline
}
\newcommand{\Poperator}[0]{-i \hbar \partial_x}
\renewcommand{\exp}[1]{\,\text{exp} \big\{ #1 \big\} }

\renewcommand{\RAISE}[0]{\hat{a}^{\dagger}}
\renewcommand{\LOWER}[0]{\hat{a}}


\begin{document}

%%%%%%%%%%%%%%%%%%
%%% HOMEWORK 1 ###
%%%%%%%%%%%%%%%%%%
\assignment{1}
Want to show 
\( \LOWER(\RAISE)^n = (\RAISE)^n\LOWER + n(\RAISE)^{n-1} \)
for $n=0,1,2,...$

\subsection*{Base Case, n=0}
\begin{align*}
    \LOWER(\RAISE)^0
    &= \LOWER \mathbb{1} \\
    &= \mathbb{1} \LOWER \\
    &= (\RAISE)^0 \LOWER + 0 \cdot (\RAISE)^{0-1}
\end{align*}

\subsection*{Induction}
Suppose that
\( \LOWER(\RAISE)^n = (\RAISE)^n\LOWER + n(\RAISE)^{n-1} \)
holds for all integers $0,1,\ldots,n$.
Need to show
\[ \LOWER(\RAISE)^{n+1} = (\RAISE)^{n+1}\LOWER + (n+1)(\RAISE)^n \]

\begin{align*}
    \LOWER(\RAISE)^{n+1}
    &= \big[ \LOWER(\RAISE)^n \big] \RAISE \\
    &= \big[ (\RAISE)^n\LOWER + n(\RAISE)^{n-1} \big] \RAISE
        && \text{by assumption}\\
    &= (\RAISE)^n \LOWER \RAISE + n (\RAISE)^n \\
    &= (\RAISE)^n \big[ (\RAISE)\LOWER + \mathbb{1} \big] + n(\RAISE)^n
        && \text{since }[\LOWER,\RAISE] = \mathbb{1} \\
    &= (\RAISE)^{n+1} + (\RAISE)^n + n(\RAISE)^n \\
    &= (\RAISE)^{n+1} + (n+1)(\RAISE)^n \\
\end{align*}



%%%%%%%%%%%%%%%%%%
%%% HOMEWORK 2 ###
%%%%%%%%%%%%%%%%%%
\assignment{2}
Find \( \bra{0} x^4 \ket{0} \)
\begin{align*}
    x = \frac{1}{\sqrt{2}} \big( 
        &\LOWER + \RAISE
        \big) \\
    x^2 = \frac{1}{2} \big( 
        &\LOWER\LOWER + \LOWER\RAISE + \RAISE\LOWER + \RAISE\RAISE
        \big) \\
    x^3 = \frac{1}{2\sqrt{2}} \big( 
        & \LOWER\LOWER\LOWER + \LOWER\RAISE\LOWER + \RAISE\LOWER\LOWER + \RAISE\RAISE\LOWER
        + \LOWER\LOWER\RAISE + \LOWER\RAISE\RAISE + \RAISE\LOWER\RAISE + \RAISE\RAISE\RAISE
        \big) \\
    x^4 = \frac{1}{4} \big( 
        & \LOWER\LOWER\LOWER\LOWER + \LOWER\RAISE\LOWER\LOWER + \RAISE\LOWER\LOWER\LOWER + \underline{\RAISE\RAISE\LOWER\LOWER}
        + \LOWER\LOWER\RAISE\LOWER + \underline{\LOWER\RAISE\RAISE\LOWER} + \underline{\RAISE\LOWER\RAISE\LOWER} + \RAISE\RAISE\RAISE\LOWER \\
        &+ \LOWER\LOWER\LOWER\RAISE + \underline{\LOWER\RAISE\LOWER\RAISE} + \underline{\RAISE\LOWER\LOWER\RAISE} + \RAISE\RAISE\LOWER\RAISE
        + \underline{\LOWER\LOWER\RAISE\RAISE} + \LOWER\RAISE\RAISE\RAISE + \RAISE\LOWER\RAISE\RAISE + \RAISE\RAISE\RAISE\RAISE
        \big)
\end{align*}
Only the underlined terms above will give non-zero inner products since they have an equal number of raising and lowering operators.
All other terms will have inner products $\braket{m|n}, m\neq n$.
The underlined terms are calculated for arbitrary $n$.
\begin{align*}
    \bra{n} \RAISE\RAISE\LOWER\LOWER \ket{n}
        &= \sqrt{n} \bra{n} \RAISE\RAISE\LOWER \ket{n-1}
         = \sqrt{n(n-1)} \bra{n} \RAISE\RAISE \ket{n-2}
         = \sqrt{n(n-1)^2} \bra{n} \RAISE \ket{n}
         = \sqrt{n^2(n-1)^2} \braket{n|n} \\
        &= n(n-1) \\
    \bra{n} \LOWER\RAISE\RAISE\LOWER \ket{n}
        &= \sqrt{n} \bra{n} \LOWER\RAISE\RAISE \ket{n-1}
         = \sqrt{n^2} \bra{n} \LOWER\RAISE \ket{n}
         = \sqrt{n^2(n+1)} \bra{n} \LOWER \ket{n+1}
         = \sqrt{n^2(n+1)^2} \braket{n|n} \\
        &= n(n+1) \\
    \bra{n} \RAISE\LOWER\RAISE\LOWER \ket{n}
        &= \sqrt{n} \bra{n} \RAISE\LOWER\RAISE \ket{n-1}
         = \sqrt{n^2} \bra{n} \RAISE\LOWER \ket{n}
         = \sqrt{n^3} \bra{n} \RAISE \ket{n-1}
         = \sqrt{n^4} \braket{n|n} \\
        &= n^2 \\
    \bra{n} \LOWER\RAISE\LOWER\RAISE \ket{n}
        &= \sqrt{n+1} \bra{n} \LOWER\RAISE\LOWER \ket{n+1}
         = \sqrt{(n+1)^2} \bra{n} \LOWER\RAISE \ket{n}
         = \sqrt{(n+1)^3} \bra{n} \LOWER \ket{n+1}
         = \sqrt{(n+1)^4} \braket{n|n} \\
        &= (n+1)^2 \\
    \bra{n} \RAISE\LOWER\LOWER\RAISE \ket{n}
        &= \sqrt{n+1} \bra{n} \RAISE\LOWER\LOWER \ket{n+1}
         = (n+1)^2 \bra{n} \RAISE\LOWER \ket{n}
         = (n+1)\sqrt{n} \bra{n} \RAISE \ket{n-1}
         = n(n+1) \braket{n|n} \\
        &= n(n+1) \\
    \bra{n} \LOWER\LOWER\RAISE\RAISE \ket{n}
        &= \sqrt{n+1} \bra{n} \LOWER\LOWER\RAISE \ket{n+1}
         = \sqrt{(n+1)(n+2)} \bra{n} \LOWER\LOWER \ket{n+2}
         = (n+2)\sqrt{n+1} \bra{n} \LOWER \ket{n+1}
         = (n+1)(n+2) \braket{n|n} \\
        &= (n+1)(n+2) \\
\end{align*}
Plugging in $n=0$ gives:
\begin{align*}
    \bra{0} x^4 \ket{0}
    &= 0(0-1) + 0(0+1) + 0^2 + (0+1)^2 + 0(0+1) + (0+1)(0+2) \\
    &= 1 + 2 \\
    &= 3
\end{align*}



\end{document}
